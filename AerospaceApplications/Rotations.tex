\documentclass[a4paper, 12pt]{article}
\usepackage{graphicx} % Required for inserting images
\usepackage[T1]{fontenc}
\usepackage[latin1]{inputenc}
\usepackage{glossaries}
\usepackage{graphicx}
\usepackage{amsfonts}
\usepackage{pifont}
\usepackage{eufrak}
\usepackage{amssymb}
\usepackage{listings}
\usepackage{verbatim}
\usepackage{tikz}
\usetikzlibrary{shapes,arrows}
\usepackage{tikz}
\tikzstyle{mybox} = [draw=black, thin, rectangle, rounded corners, inner ysep=5pt, inner xsep=5pt, fill=blue!15]
\newtheorem{theorem}{Teorema}
\usepackage[a4paper, top=1cm , bottom=1.5cm , right=1cm , left=1cm ]{geometry}

\title{
    \vspace{-1.5cm}
    \textbf{Math tools for ROTATIONS (Formulary)}
    \vspace{-0.3cm}
    }
\author{
    \textsf{\textbf{Carlo MIGLIACCIO (S332937)}}}
\date{}

\begin{document}
\maketitle

\vspace{-1.5cm}
\section*{Direction Cosine Matrices (DCM)}
{\small{

\textbf{Position of a particle}  \quad
 $ \begin{aligned}
        &\mathbf{R} = X\mathbf{I} + Y \mathbf{J} + Z \mathbf{K}
        &\text{Position of the particle in F1}\\
        &\mathbf{r} = x \mathbf{i} + y \mathbf{j} + z \mathbf{k}
        &\text{Position of the particle in F2}\\
        &\mathbf{R_O}=X_O \mathbf{I} + Y_O \mathbf{J} + Z_O \mathbf{K}
        &\text{Position in F1 of the O of F2}
    \end{aligned}$

  \noindent
\textbf{F1 vs F2} \quad
    $\begin{bmatrix}
        X\\Y\\Z
    \end{bmatrix} = 
    \begin{bmatrix}
        X_O\\Y_O\\Z_O
    \end{bmatrix}+
    \begin{bmatrix}
        \mathbf{I \cdot i}&\mathbf{I \cdot j}&\mathbf{I \cdot k}\\
        \mathbf{J \cdot i}&\mathbf{J \cdot j}&\mathbf{J \cdot k}\\
        \mathbf{K \cdot i}&\mathbf{K \cdot j}&\mathbf{K \cdot  k }
    \end{bmatrix}
    \begin{bmatrix}
        x\\y\\z
    \end{bmatrix}, \mathbf{T} \doteq \begin{bmatrix}
        \mathbf{I \cdot i}&\mathbf{I \cdot j}&\mathbf{I \cdot k}\\
        \mathbf{J \cdot i}&\mathbf{J \cdot j}&\mathbf{J \cdot k}\\
        \mathbf{K \cdot i}&\mathbf{K \cdot j}&\mathbf{K \cdot  k }
    \end{bmatrix}$

    \noindent
    \textbf{Interpretations of DCM}
    $\begin{bmatrix}
        \ X \ \\ \ Y \ \\\ Z \
    \end{bmatrix} = 
    \mathbf{T} 
    \begin{bmatrix}
       \ x \ \\ \ y \ \\ \ z \
    \end{bmatrix} \textsf{can be seen as: }
    \begin{cases}
        \text{Coordinate transformation} & \textsf{F2}\to\textsf{F1}\\
        \text{Rotation} & \textsf{F1}\to\textsf{F2}
    \end{cases} 
    $
}}

    \vspace{-0.5cm}
\section*{Euler angles}
{\small{



\textbf{Elementary 3D-rotation matrices} 
$
\begin{aligned}
    &\mathbf{T}_1(\phi) = 
    \begin{bmatrix}
        1&0&0\\
        0&\cos\phi&-\sin\phi\\
        0&\sin\phi&cos\phi
    \end{bmatrix} & \text{Rotation about $X$ (or x) of $\phi$ }\\
    &\mathbf{T}_2(\theta) = 
    \begin{bmatrix}
        \cos\theta&0&\sin\theta\\
        0&1&0\\
        -\sin\theta&0&\cos\theta
    \end{bmatrix}  & \text{Rotation about $Y$ (or y) of $\theta$ } \\
    &\mathbf{T}_3(\psi) = \begin{bmatrix}
        \cos\psi&-\sin\psi&0\\
        sin\psi&cos\psi&0\\
        0&0&1
    \end{bmatrix}
    & \text{Rotation about $Z$ (or z) of $\psi$ }\\
\end{aligned}
$

\noindent
\textbf{Rotations as product of the matrices $\mathbf{T}_\diamond$}
\begin{itemize}
    \itemsep0em
    \item[\ding{70}] \textbf{6 Tait-Bryan rotations} the most used are 123 and 321
    \item[\ding{70}] \textbf{6 proper Euler rotations} the most common is 313
\end{itemize}

}}

\vspace{-0.5cm}
\section*{Angle-axis representation $\mathbf{T} \equiv \mathbf{T}(\beta, \mathbf{u})$}
{
    \small{
\textbf{\underline{Theorem (angle-axis representation)}}
            \begin{itemize}
                \itemsep0em
                \item[(i)] Any rotation of a rigid body where a point is fixed is \textbf{equivalent} to a rotation in which the rotation axis passes through the fixed point; 
                \item[(ii)] The rotation axis is the \textbf{eigenvector} $\mathbf{u}$ corresponding to the eigenvalues 1 of the rotation matrix. 
                \vspace{0.2cm}
            \end{itemize} 
}
}

\vspace{-1cm}
\section*{\color{black}{Quaternions}}
{\small{



\textbf{Basis} \ $\{1, \mathbf{i, j, k}\}$\\
{
\centering
\textbf{Properties} \ 
$\begin{aligned}
    &\mathbf{i}^2=\mathbf{j}^2=\mathbf{k}^2=\mathbf{i\otimes j \otimes k}=-1\\
    &\mathbf{i}\otimes\mathbf{j}=-\mathbf{j}\otimes\mathbf{i}=\mathbf{k}\\
    &\mathbf{j}\otimes\mathbf{k}=-\mathbf{k}\otimes\mathbf{j}=\mathbf{i}\\
    &\mathbf{k}\otimes\mathbf{i}=-\mathbf{i}\otimes\mathbf{k}=\mathbf{j}
\end{aligned}$
\textbf{Notations} \ 
$\begin{aligned}
    \mathfrak{q} \quad &= \quad q_0+q_1\mathbf{i}+q_1\mathbf{j}+q_2\mathbf{k}=\\
                 &= \quad q_0+\mathbf{q}=\\
                 &=\quad (q_0, q_1, q_2, q_3)=\\
                 &=\quad  (q_0, \mathbf{q})=\begin{bmatrix}
                    \ q_0 \ \\\ \mathbf{q} \
                 \end{bmatrix}
\end{aligned}$
}\\
\textbf{Null element}\ $\mathfrak{O} = (0, \mathbf{0})$ \quad
\textbf{Complex conjugate} $\mathfrak{q}^*=q_0-q=\begin{bmatrix}
    q_0\\-\mathbf{q}
\end{bmatrix}=(q_0, -\mathbf{q})$
\textbf{Identity} \quad
$ \mathfrak{I} = (1, \mathbf{0})$\\
\noindent\\
\textbf{Quaternion norm}
$\vert \mathfrak{q} \vert = \Vert \mathfrak{q} \Vert = \vert \mathfrak{q}^* \vert=\sqrt(q \cdots q^* )=
\sum_{i=0}^3 {q_i}^2$ \quad
\textbf{Reciprocal quaternion}
$ \mathfrak{q}^{-1}=\frac{\mathfrak{q}^*}{\Vert \mathfrak{q} \Vert}$\\

\noindent
\textbf{Sum}
$ \mathfrak{q}+\mathfrak{p}=q_0+p_0+\mathbf{q}+\mathbf{p}$
\quad \quad
\textbf{Dot product}
$\mathfrak{q} \cdot \mathfrak{p} = 
\sum_{i=0}^3 {q_i p_i}$\\

\noindent
\textbf{Hamilton product}
$\begin{aligned}
    \mathfrak{q} \otimes \mathfrak{p} = 
    (q_0+\mathbf{q}) \otimes (p_0+\mathbf{p}) = (...)
    = (q_0p_0-\mathbf{q}\cdot\mathbf{p})+(q_0\mathbf{p}+p_0\mathbf{q}+\mathbf{q}\times\mathbf{p})
\end{aligned}$\\

\noindent
\textbf{Quaternion associated with rotations}
$\mathfrak{q} \doteq 
\bigg( 
    \cos\frac{\beta}{2}, \
    \mathbf{u} \sin\frac{\beta}{2} \
\bigg)=
\bigg( 
    \cos\frac{\beta}{2}, \
    u_1 \sin\frac{\beta}{2}, \
    u_2 \sin\frac{\beta}{2},\
    u_3 \sin\frac{\beta}{2} \
\bigg)$\\
\noindent
\textbf{Theorem } $ (0,p) = \mathfrak{q} \otimes (0, \mathbf{r}) \otimes \mathfrak{q}^*$\\
\noindent
\textbf{Elementary transformations}
$
\begin{aligned}
    &\mathbf{T}_1(\phi) \longleftrightarrow
    \mathfrak{q}_1 = \bigg(\cos \frac{\phi}{2}, \sin \frac{\phi}{2}, 0, 0 \bigg)\\
    &\mathbf{T}_2(\theta) \longleftrightarrow 
    \mathfrak{q}_2=\bigg(\cos \frac{\theta}{2}, 0, \sin \frac{\theta}{2}, 0 \bigg)\\
    &\mathbf{T}_3(\psi) \longleftrightarrow 
    \mathfrak{q}_3=\bigg(\cos \frac{\psi}{2}, 0, 0, \sin\frac{\psi}{2} \bigg)
\end{aligned}
$\\
\textbf{Any rotation can be expressed as: }
$ \mathfrak{q} = 
\mathfrak{q}_1 \otimes
\mathfrak{q}_2 \otimes
\dots \otimes
\mathfrak{q}_n$ \\

\noindent
\textbf{Inverse rotation} 
$ \mathfrak{q}^{-1} = \mathfrak{q}^*$

}}

\end{document}